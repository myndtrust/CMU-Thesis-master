% \sisetup{
%   round-mode          = places, % Rounds numbers
%   round-precision     = 2, % to 2 places
% }


\begin{table*}[ht!]
  \begin{center}
    \scalebox{0.7}{%
    \pgfplotstabletypeset[
      multicolumn names, % allows to have multicolumn names
      col sep=comma, % the seperator in our .csv file
      display columns/0/.style={
		column name=\textbf{Site}, % name of first columm
		string type},  % use siunitx for formatting
      display columns/1/.style={
		column name=\textbf{German},
		column type={S},string type},
	  display columns/2/.style={
		column name=\textbf{English},
		column type={S},string type},
	  display columns/3/.style={
		column name=\textbf{Spanish},
		column type={S},string type},
	  display columns/4/.style={
		column name=\textbf{French},
		column type={S},string type},
      display columns/5/.style={
		column name=\textbf{Japanese},
		column type={S},string type},
	  display columns/6/.style={
		column name=\textbf{Russian},
		column type={S},string type},
	  display columns/7/.style={
		column name=\textbf{Chinese},
		column type={S},string type},
	  display columns/8/.style={
		column name=\textbf{Total},
		column type={S},string type},
      every head row/.style={
		before row={\toprule}, % have a rule at top
		after row={
% 			\si{\ampere} & \si{\volt}\\ % the units seperated by &
			\midrule} % rule under units
			},
		every last row/.style={before row={\toprule}, after row=\bottomrule}, % rule at bottom
    ]{embodied_cost_model/contents/data/ops_co2_table3.csv} % filename/path to file
  }
  \end{center}
     \caption[Data Center Operation Carbon Footprint]{Data Center Operation Carbon Footprint (Tonnes of CO2 Emitted over the Year)}
    \label{dc_carbon_footprint_mec}
\end{table*}