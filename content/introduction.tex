\section{Introduction}

Data centers play a critical role in shaping modern economies and the lives of people around the world. They house information technology (IT) hardware and store data for financial institutions, social media accounts, and virtual meeting forums amongst many other things. In their physical embodiment, networks of data centers are global scale systems. Within these systems, each data center’s size can be as large as a college campus, and consume 100 times the amount of power every hour as residential facilities do in a day. Moreover, compared to residential facilities, data centers are subject to bulks of accelerated turnovers in technology. This technology turnover results in the housed IT equipment having a useful life that only spans two to three years. The short useful life of equipment increases the ecological footprint of these systems, due to all the materials that traverse through it over its lifetime. Therefore, it is extremely important to develop a comprehensive global level modeling framework to assess data center ecological footprints. 

Today there are several metrics, models, and methods available that allow us to assess energy use of data center systems. At the building scale, the most prevalent metric is the data center power usage effectiveness (PUE). PUE couples the IT loads with the building and indicates the relative efficiency of the building utilities compared to IT workloads. For PUE calculations, the IT workloads are represented in terms of power demand of the IT equipment. 

However, such an approach neglects the power inefficiencies associated with hardware and the sustainability of the source energy powering the data center. An exception to just energy use modeling is the CLEER online tool. CLEER provides data center level models that account for all the materials that data center systems are composed of. Yet, CLEER’s publicly accessible capability does not allow for modeling of dynamic workloads that are the characteristic of internet services, nor does it support explicit building energy models. A footprint accounting for all the carbon embodied in materials and in the operational energy will indicate the global warming potential attributed to the respective data center system.

To fill the gaps noted above, I demonstrate a comprehensive data center energy and carbon footprint modeling framework which is discussed in the four chapters of this dissertation. In the first chapter, I develop a simulator that characterises internet traffic to five internationally dispersed data centers for seven real world internet services. I implement the simulation to indicate the workload demand from each service to each data center. In the second chapter, I take the data center workloads and construct them with a physics based building energy modeling tool, EnergyPlus. In EnergyPlus the traffic is used as coefficients to the IT load profiles that each data center must support for every coincident hour of a year. Then in the third chanter, I take the results from the EnergyPlus model and quantify the marginal costs of the energy in terms of carbon footprint; where the respective data center’s regional power grid consists of a mix of renewable and carbon based energy sources. Finally in the fourth chapter, I couple the marginal energy costs model with an environmentally extended economic input-output model that quantifies the environmental costs embodied in the materials that the data center is composed of. 

Through this dissertation I contribute a comprehensive model that quantifies the end to end carbon footprint of global data center systems. Researchers and data center operators can use my demonstrated framework for evaluating current data center global warming potential or to make strategic decisions on how to deploy future data centers.

 
