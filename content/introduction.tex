\section{Introduction}

Data centers play a critical role in shaping modern economies and lives of people around the world. They house information technology (IT) hardware and store data for financial institutions, social media accounts, and virtual meeting mediums amongst many other things. In their physical embodiment, data centers are global scale systems. In these systems, each data center’s size can rival college campuses and consume 100 times the amount of power every hour as residential facilities do in a day. Moreover, compared to residential facilities, data center’s are subject to bulks of accelerated churn in technology. The technology churn results in the IT equipment having a useful life that only spans two to three years. The short useful life of equipment increases the ecological footprint of these systems, due to all the materials that traverse through it over its lifetime. Therefore it is of great importance to develop a comprehensive global level modeling framework to assess the system level ecological footprint that accounts for all of a data center system’s materials and operational energy.

Today there are several metrics, models, and methods available that allow assessment of energy use of data center sub systems. At the building scale the most prevalent metric is the data center power usage effectiveness (PUE). PUE couples the IT loads and the building and is an indicator of the relative efficiency of the building utilities and IT workloads. For PUE calculations, the IT workloads are represented in terms of power demand of the IT equipment. However, such an approach neglects the power inefficiencies associated with hardware. 

An exception to just energy use modeling is the CLEER online tool. CLEER provides data-center level models that account for all the materials that data center systems are composed of. However, CLEER’s publicly accessible capability does not extend to allow for dynamic workloads that are the characteristics of internet work loads, nor does it support explicit building energy models. 

In this dissertation, I demonstrate a comprehensive data center energy and carbon footprint model that consists of four parts. First, I develop a simulator that characterises internet traffic to five internationally dispersed data centers for seven real world internet services. I implement the simulation to indicate the workload demand from each service to each data center. Second, I take the data center wise workloads and construct them with a physics based building energy modeling tool, EnergyPlus, as the coefficients to the IT load profiles that each data center must support every hour of a year. Third, I take the results from the EnergyPlus model and quantify the marginal costs of the energy in terms of Carbon footprint; where the respective region’s power grid consists of a mix of renewable and carbon based energy sources. Fourth, I couple the marginal energy costs model with a model that quantifies the environmental costs embodied in the materials that the data center is composed of. Through this dissertation I contribute a comprehensive model that quantifies the end to end carbon footprint of global data center systems. 

Researchers and data center operators can use my demonstrated framework for evaluating current data center environmental footprints or to make strategic decisions on how to deploy future data centers. 
