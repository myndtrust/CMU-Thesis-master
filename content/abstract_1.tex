\section{Abstract}

Data centers (DCs) are critical for modern society. They house information technology (IT) hardware and store data for financial institutions, social media accounts, entertainment portals, and virtual meeting forums amongst many other things. In their physical embodiment, networks of DCs are global scale systems. Within these systems, each DC’s size can be as large as a college campus and consume 100 times the amount of electrical energy every hour as residential facilities do in a day. Given their growing demand, it is extremely important to develop a global level modeling framework to help make DC design decisions that comprehensively account the total costs of ownership (TCO) of DCs inclusive of capital and operational costs..

Currently, DC design practices optimize for the power usage effective metric. However, some industry insiders have expressed that thinking of PUE in isolation may lead to inadvertently increasing the TCO for DC systems. To fill the isolated view of PUE, this research developed a prototype of an agile model that uses life cycle analysis (LCA) frameworks to quantify the holistic life cycle cost of DC systems. The developed model is composed of four software modules: 

\begin{itemize}
\item A real world internet traffic profile simulation module as a proxy for DC workloads. 
\item A module that integrates building energy simulations with dynamic DC workloads in a novel way.
\item A module that couples the building energy demands with the marginal costs of energy production.
\item A module that is composed of a hybrid of a process and economic input-output based LCA frameworks to assess the embodied costs of the global DC system.
\end{itemize}
As a framework for comprehensive TCOs, this dissertation provides DC designers two contributions. The first contribution is the affirmation that characterising DC costs requires a modular approach. This modular approach is at least three tiered, where the first tier must characterise the infrastructure in terms of materials and operational processes. In the second tier, the model must be aware of the workload interactions within the system. The third and final tier allows the application of objective models to quantify various forms of costs associated with DCs. Using this three tiered approach a demonstration for the total carbon footprint of a set of hypothetical internet service is demonstrated.   

The second contribution is shown through the coupling of the network driven workloads and the building energy simulations. The research proves the feasibility of inverse cooling plant controls; where the chiller operational point can be kept at a constant load by varying the IT power loads for batch tasks. Opportunistic varying of batch tasks allows over-subscription of workloads when the cooling plant is at conventional part loads.