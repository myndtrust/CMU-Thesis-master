\chapter{Methodology for Overall Research Strategy}

\section{Chapter Abstract}

This chapter describes the methods used to identify the scope of this dissertation's research around the total costs of ownership of data centers, the methods used for articulating a succinct research question for such a broad and emerging area application, and the the methods used to execute the research. It begins by describing data centers to set the context of the research undertaking. Then the chapter provides the methods that led to developing a model that is capable of quantifying the end to end costs of a the global system using life cycle analysis frameworks by demonstrating it effectiveness with the quantification $CO_2$ emissions. 

\section{Introduction}

Data centers (DCs) will consume 13\% of the worlds energy production by 2030 according to worst case prediction models \cite{andrae15}. This high energy demand is not surprising given the ubiquitous penetration of the internet nowadays. For internet DC infrastructure, the energy demand problem space is compounded by a rapid pace of hardware technology innovations on one hand. On the other hand, factoring software technology innovations lead to models that show a brighter picture. This picture is aligned with current trends, where efficiency across the entire DC stack indicates a downward trend for power demand per unit of performance. As an example, by 2016 performance throughput had increased 350\% since 2012, relative to the power demand for computer hardware \cite{GoogleEnvRpt}.  The hardware efficiency gains seen over the last decade still do not offset the need for additional DC physical capacity.

Internet DC capacity is not bound by physical sizes or geographic locations however. Location agnostic networks spanning the globe allow information technology (IT) systems to scale in resource parameters such as CPU, GPU, memory, storage, and data access rates. Parametric scaling of these resources enables operational tuning to compensate for physical capacity constraints. Capacity fungibility of DC parameters at a global scale leads businesses with risky practices to provision building scale data-center infrastructure very lean. While businesses that are adverse to risk can be excessively conservative. In the case of lean provisioning, all physical resources are highly utilized with little or no headroom. For the conservative approaches, normal operating conditions are such that sustained operational points are well below the capacity of the equipment. Operations significantly below capacity allows the service to accept bursts in service, inorganic  growth, and be fault tolerant in case of failures elsewhere in the system. Lean and conservative operations are both classified as equivalent based on today's sustainability indicators.

The prevalent indicator for sustainability of DC facilities is currently the Power Usage Effectiveness metric ($PUE$, see equation \ref{eq:pue}). $PUE$ is a dimensionless ratio of load vs. supply energy.   In the equation $E_{IT}$ indicates the IT energy load and $E_{total}$ indicates the total energy supplied to the facility. The $PUE$ is generally reported in a quarterly or annual basis by integrating the equation across the respective time period. For a given period, the metric measures the operational efficiency of the facility's system compared to the information technology equipment (ITE). Spaces housing ITE have seen profound cooling and power distribution efficiency gains resulting from the focus on $PUE$, however the metric does not comprehensively cover sustainability of DCs.

\begin{equation} \label{eq:pue}
PUE=\frac{E_{total}}{E_{IT}} 
\end{equation}

The $PUE$'s shortcomings as a sustainability metric are noted by Horner, particularly that low PUEs do not correlate to low carbon footprint \citep{Horner16}. The tendency of carbon intensity to vary based on energy source characteristics is demonstrated in \citep{Masanet13a}. Energy supply sources, when coupled with the demands are documented to be good indicators of use phase carbon footprint. For example, the carbon footprint of a low $PUE$ DC with it's energy sourced from a coal plant will be higher compared to a DC that operates with a slightly higher $PUE$ but sources power from a renewable source. 

Beyond the operations phase of DCs, the IT hardware and physical infrastructure products they're composed of have many other life-cycle phases. At each life cycle phase of the products there are energy and $CO_2$ inventories. Inventories when tallied from the useful state (techno-sphere\footnote{Techno-sphere: processes and products transformed from a raw state to a useful form by human intervention.} augmented) of the product to their natural state (raw form in the bio-sphere) constitute the comprehensive life cycle perspective.  The perspective of this work classifies inventories as either a debt or credit to it's environmental footprint. The debt and credit approach follows the spanning tree structure analogous to total costs of ownership (TCO) methods used in today's sophisticated business models for DCs. The modern business models amortize capital and operating costs over a system's useful life similar to the treatment of energy and $CO_2$ inventories in this work.

The direct effects described so far and discussed throughout this work are the most transparent impacts associated with information communication technology (ICT). The indirect or higher order impacts fundamentally alter the use of energy in a wide breadth of applications. Figure \ref{ICT_econ} shows examples of the breadth ICT impacts to other industrial sectors. However, this proposal will follow the approach set by The Green Grid \citep{tgg12} where the DC is a physical entity of interest and focus on the assessment of its life cycle activities. This is indicated by the red boundary shown in figure  \ref{fig:f2} and sets the framework for the boundary conditions established later. 