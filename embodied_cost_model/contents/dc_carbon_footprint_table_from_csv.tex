\sisetup{
  round-mode          = places, % Rounds numbers
  round-precision     = 0, % to 2 places
}


\begin{table}[h!]
  \begin{center}
    \label{dc_carbon_footprint}
    \scalebox{0.5}{%
    \pgfplotstabletypeset[
      multicolumn names, % allows to have multicolumn names
      col sep=comma, % the seperator in our .csv file
      display columns/0/.style={
		column name=Site, % name of first colum
		string type},  % use siunitx for formatting
      display columns/1/.style={
		column name=German,
		column type={S},string type},
	  display columns/2/.style={
		column name=English,
		column type={S},string type},
	  display columns/3/.style={
		column name=Spanish,
		column type={S},string type},
	  display columns/4/.style={
		column name=French,
		column type={S},string type},
      display columns/5/.style={
		column name=Japanese,
		column type={S},string type},
	  display columns/6/.style={
		column name=Russian,
		column type={S},string type},
	  display columns/7/.style={
		column name=Chinese,
		column type={S},string type},
	  display columns/8/.style={
		column name=Total,
		column type={S},string type},
      every head row/.style={
		before row={\toprule}, % have a rule at top
		after row={
% 			\si{\ampere} & \si{\volt}\\ % the units seperated by &
			\midrule} % rule under units
			},
		every last row/.style={before row={\toprule}, after row=\bottomrule}, % rule at bottom
    ]{embodied_cost_model/contents/data/ops_co2_table2.csv} % filename/path to file
  }\end{center}
     \caption[Data Center Operation Carbon Footprint]{Data Center Operation Carbon Footprint}
\end{table}