\chapter{Wide Area Network Based Data Center Energy Simulations for Internet Services}
\label{chp:traffic}

\section{Chapter Abstract}
Data centers are a critical component of modern day economies and social lives of people all around the world. In many aspects society’s dependency on internet services enabled by data centers is still increasing; the COVID-19 stay in place orders are a current example of unprecedented growth rates for many internet services. With the increased dependency on data centers, there is a concern about their environmental impacts across at the globe scale. To reason about the environmental footprint of these globally spanning system requires a new approach to model building energy use of facilities within these systems. This paper demonstrates a top down process that leverages publicly available network traffic data for a globally distributed internet site to model it’s power demand, using Wikipedia as the model internet site. The traffic provides an indication of the building loads that the data center infrastructure must support.  

This chapter demonstrates a method to geographically correlate five data centers with the incoming visits from a global user base using a minimum cost function. The correlations are further extended to provide a forecast of the traffic. The contribution of this research is two-fold. First, it presents a method for a top down assessment of network traffic that can characterize the information technology loads in building energy models. Second, the traffic forecast allows data center operators and designers to model future demands of the building infrastructure systems and perform scenario models based on the projected traffic. 

\section{Introduction}

Data centers are ubiquitous in today’s society. Consumers of data center enabled web services expect high availability of their functionality and access to their data on a continual basis. These expectations are met with software architectures that provide strongly consistent views of the data regardless of location and in the presence of physical systems failures. In terms of specific failure domains, network failures can be the most catastrophic. For example, an outage of a network link connecting a data center can lead to all physical resources in it to become inaccessible from the outside. With these expectations, the dependencies of internet data centers and their communications network during failures is clearly apparent.

Data center utilities (such as mechanical cooling resources and power distribution resources) and their network connections are not only coupled in failure scenarios described above, but they are also in lock-step with each other through normal operations. This coupling is also easy to reason about; as internet data centers don’t generate workloads hermetically. The workload demands for data centers come from remote consumers (user facing traffic) and other adjacent data centers (back-end traffic). Both forms of incoming traffic instantiate computational and storage processes on the IT hardware, which in turn are powered and cooled by district scale distribution plants.

In the current generation of data centers, the building utility systems demand less than 10\% of the total energy delivered to the site \cite{Shehabi16}. The balance of the power of these new data centers is consumed by the information technology (IT) equipment. With this disproportionate allocation of energy, it is clear that the total energy use in data centers is more sensitive to ITE loads than the facilities systems. Yet the building systems still receive the most attention in the design and facilities operations phase of the buildings. Others have also pointed out similar gaps data center building energy models \cite{Beatty15}. 

To provide an intuition of data center networks, Figure 1 shows a generalized depiction their topology. At the top are the internet service provider (ISP) networks are shown as clouds because the physical paths of these systems are managed by the global ISPs and they are typically out of control of data center operators. Data center operator’s depend on the ISPs for connecting their facilities through private wide area networks (WAN). The pink lines indicate these wide area connections that traverse over ISP networks. The magenta links connect global WANs data center metropolitan, which typically are responsible for the last miles of network distribution. Within the metropolitan area, their maybe one or more data center sites, only two are shown for clarity in the figure.  